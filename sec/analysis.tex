% !TeX spellcheck = en_GB
\begin{document}
   As the purpose of the implementation was to know if the cryptosystem was
   good for production, a test of the cost in time for sending packages was
   needed. This results, given in the \ref{ana:res}, made it clear that
   the system is capable of holding at least 128 smart meters. Even though,
   there are still some points that the test fails to show.\\
   \\
   The first point is the problems in the connectivity of the aggregated system, as the test were done in a single machine, using the interface of localhost.
   The second point refers to the number of meters and the problem of simulating the infrastructure in a single machine. As the machine used for testing only has available two cores for processing, when using it for 129 threads for simulating (128 smart meters and 1 substation), the time of switching between process skyrockets.
   
\end{document}
