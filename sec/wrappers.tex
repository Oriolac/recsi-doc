% !TeX spellcheck = en_GB
\begin{document}
	% Introduction to wrappers and algorithms [references]
	% Why made this three classes.
	% Problem of Pollard's Lambda and Hashed Force
	The cryptosystems described in \cite{recsi, busom} need to solve a discrete logarithm
	problem, so a module for algorithms was made at the library.
	Three algorithms was used to test them. At first it was used Pollard's
	Lambda, but a bug was found out that made it sometimes fail. Even though, 
	 the same point, when used in unit testing, worked flawlessly. To advance the
	implementation, a Brute Force algorithm was created and tested.
	It worked, so we started to gather other methods to solve the problem.\\
	\\
	The third algorithm was a mesmerized brute force. As the cost in memory was
	doable -$2^{20}$ instances are proximately 1 GB in memory-, every instance was saved. At this point, the bug that was in Pollard's
	Lambda was found to be that the hashcode and equals methods were not consistent, so both methods were remade in all CigLib. It was made an improved version that does not mesmerize all instances, instead it does every $times$ instances.
	\subsubsection{Times}
	\begin{table}[h]
		\centering
	\begin{tabular}{l|rrrr}
		Algorithm &  Mean (ms) & Median (ms) & Max (ms) &Min (ms)\\ 
		\hline 
		Pollard's Lambda&112.21&99.5&205&92  \\
		Brute Force &1576.36 &1508.0 &716 &2380\\
		Hashed &18.29 &17.5 &12 &32
		

	\end{tabular}
	\caption{Time comparison of all algorithms}
	\label{wrap:time}
	\end{table}
	As seen in table \ref{wrap:time},
	the best algorithm is observed to be the mesmerized brute force (called Hashed).
	The cost of the algorithm is only paid when the substation is set up, so it works with this use case of solving the discrete logarithm.
\end{document}